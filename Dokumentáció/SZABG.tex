\chapter{Algoritmus általános célokra}
\section{Wave function collapse - Az ötletgazda} \label{WFC-Otlet}
A "WaveFunctionCollapse" nevű algoritmust először githubra publikálta egy felhasználó \footnote{https://github.com/mxgmn/WaveFunctionCollapse}. Az algoritmus nevét a szerző a "hullámfüggvény összeomlásáról" nevezte el, ami egy kvantummechanikai kifejezés. Magának az algoritmusnak nincs szoros kapcsolata ezzel a jelenséggel, de az ihlet onnan jött.
\\Az algoritmus eredeti ötlete, hogy egy inputként megadott bitmap pixeljeit felhasználva generálunk egy, az eredetihez lokálisan hasonló általában nagyobb bitmapet. A módszer azonban nem korlátolt csak bitmapekre: egy adott képet is fel lehet fogni egy bitmapként, amit tetszőlegesen darabolhatunk fel, így lényegében bármilyen képre elvégezhetjük a lépéseket. Ez kifejezetten hasznos több területen is. Egy ilyen felhasználási módszer amit a forrás is említ, hogy az algoritmussal csempézni is lehet a teret.
\\Ezen algoritmus inspirált a sajátos megoldásom elkészítésében. Maga a WFC alapköve az, hogy bármely $NxN$-es blokk a kimenetben fellelhető valamilyen forgatással a bemenetben. Ez a szabály egy elég erős feltétel, illetve magába foglal egyfajta szomszédsági kitételt. A "rejtett" szomszédsági kitétel ihlette az én implementációmat.
\section{Szomszédság alapú blokk generálás - SZABG}
Ahhoz hogy a \ref{WFC-Otlet} pontban említett WFC ihlette algoritmus implementációm egyszerűbben hivatkozható legyen, nevezzük SZABG-nek.
A SZABG eljárás bemenetére csak annyi a kitétel, hogy egy $NxM$-es PNG formátumú kép legyen. Viszont ahhoz, hogy elfogadható kimenetet kaphassunk érdemes az alábbi pontokat szem előtt tartani:
\begin{itemize}
    \item A bemenet legyen egy bittérkép, melyben a pixelek szomsédságát adjuk meg. A pixeleket blokkokként kezelhetjük.
    \item Ha a bemenet nem egy bittérkép, akkor a kép legyen olyan blokkokból készítve, melyek szomszédsága ésszerűen elkészített \textit{(tehát például 16x16 pixelből álló bokor, fa és fű szomszédságának megadása úgy, hogy ezek nem fedik egymást és két blokk között nincs hézag \#KÉP)}.
    \item Használhatunk üres blokkokat is. Üres egy blokk, ha annak alfa csatornája 0. Ezen blokkok nem számítanak bele a szomszédsági kapcsolatokba, és a kimenetben sem szerepelnek. Szerepük a kapcsolatok kialakításának megkönnyítése \#KÉP.
    \item Ha két $NxM$-es blokkban pontosan ugyanazok a pixelek találhatóak, egynek vesszük őket. Ekkor az adott blokkhoz tartozó súly számláló értéke növekszik, aminek eredményeképpen a blokk előfordulásának valószínűsége nagyobb lesz a kimenetben.
    \item A szomszsédságok iránya nem mindegy. Ha egy A blokk csak balról szomszédja egy B blokknak, akkor csak és kizárólag balról érintheti a B blokkot a kimenetben.
    \item A bemenet beolvasása függ a paraméterezéstől, ezért érdemes összehangolni a kettőt.
\end{itemize}

\subsection{Állítható paraméterek}
A SZABG algoritmust futattó program azon paraméterei, melyek a felhasználók által állíthatóak annak futása előtt.
\subsubsection{Bemeneti / kimeneti paraméterek}
Ezen paraméterek határozzák meg, hogy az input képen mekkorák a blokkok, illetve, hogy az output kép nagyságga mekkora legyen.
\begin{table}[H]
\caption{Bemeneti / kimeneti paraméterek}
\label{bemeneti-kimeneti-szabg}
\begin{center}
\begin{tabular}{ |c|c|c| } 
\hline
Paraméter & Leírás & Érték típus \\
 \hline\hline
 Block Width & Egy blokk szélessége & Pozitív egész szám \\ 
 \hline
 Block Height & Egy blokk magassága  & Pozitív egész szám\\ 
  \hline
  Image Width & A generált kép szélessége & Pozitív egész szám \\ 
 \hline
 Image Height & A generált kép magassága  & Pozitív egész szám\\ 
  \hline
\end{tabular}
\end{center}
\end{table}

\paragrafus{Egyéb paraméterek}
Ezek a paraméterek különböző részeit állítját az algoritmusnak, nagyban befolyásolva az eredményt.

\begin{enumerate}
    \item Block Can Neighbor Itself
    \begin{itemize}
        \item Logikai érték. Alapértelmezett értéke: hamis.
        \item Igaz érték esetén globálisan minden blokk bármilyen szomszédja lehet önmagának. A súlyokat és a blokk más blokkokra vett szomszédsági kapcsolatát nem befolyásolja.
        \item Alternatívaként készíthetünk a bemenetre olyan blokkokat, melyek leírják ezt a szomszédságot. Ezzel kiválasztva, hogy pontosan mely cellák legyenek önmaguk szomszédjai. \#KÉP
    \end{itemize}
    \item Wildcard
    \begin{itemize}
        \item CellVariable típusú paraméter. Alapértelmezett értéke: null.
        \item A felhasználó ezt egy vizuális panelen tudja beállítani, ha kevesebb mint 40 különböző blokkot tartalmaz a bemenet.
        \item Egy "joker" blokként fogható fel a beállított blokk. Lényege, hogy bármilyen blokk bármilyen szomszédja lehet. Ha adtunk neki értéket, akkor az algoritmus garantáltan nem akad el.
    \end{itemize}
    \item Block global weight
    \begin{itemize}
        \item Pozitív valós szám. Alapértelmezett értéke a bemenet beolvasása során állítódik.
        \item A felhasználó ezt egy vizuális panelen tudja módosítani minden blokkhoz, ha kevesebb mint 40 különböző blokkot tartalmaz a bemenet.
        \item A paraméter befolyásolja, hogy egy adott blokk milyen globális súllyal rendelkezzen. Minél nagyobb a blokk súlya, annál inkább valószínű, hogy az input egy adott blokkja ilyen legyen (persze az előfordulás nem csak ettől függ). Lokális súlyokat használva nincs hatása.
    \end{itemize}
    \item Cell mode
    \begin{itemize}
        \item Logikai érték. Alapértelmezett értéke: hamis.
        \item Igaz érték esetén a generálás nem ténylegesen egy képet készít, hanem sok kicsit. Pontosabban annyit, amennyi blokkal lefedhető a kívánt kép. Nagyban befolyásolja a program futási idejét, viszont ebben a "cella módban" animálható az algoritmus. Az implementáció során jóval egyszerűbb ezt a módot elkészíteni először.
        \item Hamis érték esetén egy darab tényleges kép objektum generálódik. Virtuális cellákat használva, minden cella erre az egy képre másolja át a tartalmat. Nem animálható, mert a kép pixeleinek beállítasa gyors, viszont annak megjelenítése lassú. Így a futás végén jelenik meg a már kész kép. Jóval gyorsabb futási eredményt okozhat.
    \end{itemize}
    \item{Bias}
    \begin{itemize}
        \item Pozitív valós érték. Alapértelmezett értéke a bemenet beolvasása során állítódik.
        \item Lokális súlyokat használva azt határozza meg, hogy a szomszédos letett cellák mennyire domináns hatással vannak az épp vizsgált cellára nézve. Minél nagyobb az érték, annál inkább lesznek egybefüggő "szigetek" a kimeneten.
    \end{itemize}
\end{enumerate}

\paragrafus{Lokális súlyok és Bias leírása}
A lokális súlyok közvetlenül nem állíthatóak. Csak akkor van hatásuk, ha a "Use local weights" értéke igaz. A bemenet beolvasása közben a program minden blokkra egyesével kialakít egy lokális súly kapcsolatot a szomszédos blokkokkal. Egy kapcsolat a két blokk referenciájából áll, illetve, hogy mekkora ezeknek a súlya. A referenciák, illetve a szomszédsági irány nem releváns. Ha két blokk a bemeneten többször áll egymás mellett, akkor a súly egyel növekszik, új kapcsolat nem jön létre. Minél nagyobb a súlyérték, annál inkább "preferálja" egy blokk azt a szomszédot.\\
Például, ha egy A-B kapcsolatnak a súlya 10, egy A-C kapcsolatnak a súlya 12, akkor a versenyzési fázis során az A blokk a C blokkot adja vissza, mint preferált érték.\\
A versenyeztetési fázis a cellák lerakásakor indul. Lépései:
\begin{enumerate}
    \item \label{lokalis1} A környező cellák először leszűrik, hogy milyen cella kerülhet melléjük.
    \item  A leszűrt cellák visszakerülnek egy ellenőrzésre minden környező cellához, ahol egyenként kiválasztják, hogy melyik cellát preferálják (feltéve, hogy van olyan). Ha több cellát is preferált, akkor véletlenszerűen választódik egy.
    \item A preferált cellák visszakerülnek a \ref{lokalis1} pontban készített listába. Így azok a cellák amelyek preferáltak, többször szerepelnek a listában, mint azok, amelyek nem.
    \item A cellákat csoportosítjuk az előfordulások számának függvényében, új súlyokat létrehozva ezzel. A legnagyobb súlyú celláknak megszorozzuk a súlyát a Bias paraméter értékével.
    \item Az így keletkezett opciókból a súlyokat figyelembe véve véletlenszerűen választunk (a nagyobb súlyú celláknak nagyobb esélye van a kiválasztásra).
\end{enumerate}
A versenyeztetés során az opciók száma nem csökken, illetve a választási valószínűségük sem lesz 0. Ez garantálja azt, hogy változatos lesz a végeredmény, még viszonylag magas Bias érték mellett is.



\begin{table}[htbp]
\caption{Egyéb paraméterek összefoglalása}
\begin{center}
\begin{tabular}{ |c|c|c| } 
\hline
Paraméter & Leírás & Érték típus \\
 \hline\hline
  \makecell{Block Can \\ Neighbor Itself} & \makecell{Ha igaz, akkor minden blokk 
 \\szomszédja önmagának minden irányból} & Logikai érték \\
 \hline
 Wildcard &\makecell{ Olyan blokk, ami bármilyen blokk \\ bármilyen szomszédja lehet} & CellVariable \\ 
 \hline
 Block global weight & Egy blokk globális súlya  & Pozitív valós szám\\ 
  \hline
  Animate & \makecell{A generálás animálódjon-e \\ (csak cella módban!)} & Logikai érték \\
 \hline
  Cell mode & Cella mód ki/be kapcsolása. & Logikai érték \\
 \hline
  Use Local weights & \makecell{Lokális vagy globális súlyokat \\ használjon-e az algoritmus.} & Logikai érték \\
 \hline
 Bias & \makecell{Lokális súlyok esetén mennyire legyen \\ domináns a letett cellák befolyása. \\ } & Pozitív valós érték \\
 \hline
\end{tabular}
\end{center}
\end{table}



\subsection{Az algoritmus lépései}
A paraméterek áttekintése után, vessünk egy pillantást a lépésekre. Önmagában nem feltétlenül tűnik bonyolultnak az eljárás elképzelése: olvassunk be blokkokat, majd tegyük le őket kvázi véletlenül. Ennél azért bonyolultabb a tényleges működés, hiszen szomszédsági kapcsolatokat kell kialakítanunk. Ez növeli a komplexitást, illetve elválasztja a teljesen véletlen generálástól a SZABG-t.
\begin{enumerate}
    \item Beolvassuk az inputot.
    \item Elkészítjuk az input alapján a blokkokat, a \ref{bemeneti-kimeneti-szabg} paramétereknek megfelelően. A blokkokhoz globális és szükségszerűen lokális súlyokat számítunk.
    \item Minden blokkhoz hozzárendeljük a pozícióját, illetve meghatározzuk, mely blokkok szomszédosak egymással.\\
    A szomszédokat az alábbiként definiáljuk:
    \begin{itemize}
        \item Ha egy blokk koordinátái: $(x,y)$, akkor szomszédjainak koordinátái: \\
        $(x-1,y), (x+1,y), (x,y-1),(x,y+1)$
        \item A blokkokat egy tóruszon képzeljük el. Ez azt jelenti, hogy ha összesen $NXK$ blokk van, akkor az $(1,j), (j = 1,2,3...k)$ koordinátájú blokkok bal szomszédja az $(n,j), (j = 1,2,3...k)$ koordinátákon helyezkednek el. Ez a kapcsolat minden szélső blokkra hasonlóan igaz.
    \end{itemize}
    \item Lefedjük az output képet üres \textit{cellákkal}, amik egy-egy blokkot fognak megjeleníteni.
    \item Kiválasztjuk, hogy melyik üres cellát szeretnénk feltölteni. Ez a kiválasztás az alábbi eljárások egyikével történik:
    \begin{itemize}
    \label{item:modszer}
        \item Horizontálisan / vertikálisan, azaz elindulunk egy soron / oszlopon és folytonosan választjuk a cellákat.
        \item Véletlenszerűen.
        \item Kiszámítjuk minden cella \textit{entrópiáját}, majd mindig a legalacsonyabb entrópiájú cellát választjuk. A \textbf{cella entrópiája} az a mennyiség, amely megadja, hogy mennyire "bonyolult" meghatározni, hogy milyen blokkot fog tartalmazni a cella. Egy cella blokkját annál bonyolultabb meghatározni, minél több közel egyenlő valószínűségű opció közül lehet választani.\\
        Legyen a cellák halmaza $C$, a cellára letehető blokkok halmaza $B$. Egy $c\in C$ cella entrópiáját az alábbi képlettel számolhatjuk: $$E(c) = \sum_{i=1}^{n}P(b_i) * \log{1/P(b_i)}$$ ahol $n$ a $B$ halmaz elemszáma, $b_i\in B$, $P(b_i)$ pedig az adott blokk \textit{globális elhelyezési valószínűsége}. Egy $b\in B$ blokk \textbf{globális elhelyezési valószínűsége:} $$P(b) = \frac{S_b}{S_B}$$\\
        Ahol $S_b$ a $b$ blokk globális súlya, $S_B$ pedig az összes, a cellára letehető blokk globális súlyának összege.
    \end{itemize}
    \item Addig ismételjük a 3. ponttól az algoritmust, míg le nem fedtuk az outputot, vagy el nem akadtunk.\\
    Az elakadás egy probléma ami akkor jelenik meg, ha olyan üres cellához érünk, melynek szomszédjai kizárják egymást. Ekkor az algoritmust újrakezdjük, vagy leállunk. Egy ilyen elakadási helyzetre példa a \#KÉP ábra.\\
    Elakadás nem lehetséges, ha van olyan blokk, amely bármilyen másik blokk szomszédja lehet.
\end{enumerate}

\newpage

\subsection{Felhasználás}
Az algoritmus első ránézésre csak textúrákat képes generálni, viszont ez nem feltétlenül igaz. Képzeljünk el egy bemenetet, ahol háromféle blokk van: barna, zöld, sárga. Ezek a blokkokat később felfoghatjuk földként, fűként és virággként. Ha egy megfelelő inputot megadunk a programnak, legenerálhatjuk egy mező felülnézeti térképét. A letett blokkok színét és helyzetét felhasználva ebből egy háromdimenziós színteret is előállíthatunk.\\
Egy kétdimenziós színtér esetén a generálás már-már kész eredményt is adhat. Nem szabad elsiklani afölött a tény fölött, hogy a színtér készítése az eljárás végével nem kell, hogy megálljon. Az eredményt tetszőleges módon lehet exportálni és importálni, a felhasználás során az adatokat átrendezhetjük.
