\chapter{Bevezető}
A színterek kulcsfontosságú szerepet töltenek be a digitális világban. Egy film, játék, animáció nem lenne teljes egy megfelelően kialakított helyszín, pálya, háttér nélkül. Emiatt szükséges olyan módszereket kialakítani, melyek effektív eszközként szolgálnak ezen elemek létrehozására. A szakdolgozatom eszközként a procedurális generálást taglalja. \\
A színterek procedurális generálása sok mindent magába foglal. Ha egy könyv leírása alapján gyurmából elkészítünk egy csatateret, lényegében generáltunk procedurálisan egy színteret. Az elkészítendő objektum nem is igazán lényeges, csak az, hogy miként készítettük el. Jelen esetben nem valódi tárgyakat készítünk el, hanem digitális adatokat generáltatunk a számítógéppel, amit megfelelő környezetben színtérként tudunk értelmezni.\\
Több generálási módszert tárgyal a szakdolgozatom, leginkább videójátékokban használatos pályák, hátterek előállítására. A célterülettől függően két külön esetet vizsgálunk : 2 dimenziós és 3 dimenziós színtereket előállító algoritmusokat.