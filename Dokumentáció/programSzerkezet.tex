

\chapter{A programról}
Az elkészített program procedurális generálási módszereket tartalmaz. Ezek az eljárások általános felhasználási módokat mutatnak be, nem specifikusan egy alkalmazási területre korlátozódnak.\\
A program használata során több algoritmust vizsgálhat meg a felhasználó, közben különféle paramétereket próbálhat ki. Az eredményként kapott színterek felhasználására a programon belül nincs lehetőség, viszont van, ahol exportálási lehetőség is adott. A kidolgozás során törekedtem arra, hogy a metódusokat sajátos célra is fel tudja bárki használni. 

\section{Tartalom}
Az első dolog amivel találkozik a felhasználó, az egy miniatűr játék-féleség, amit az úgynevezett "Dinosaur Game" ihletett. A Dinosaur Game nevű játékot valószínűleg sokan ismerhetik, hiszen ez a játék jelenik meg a Google Chrome böngészőben, ha nincsen internetelérés.\\
Ez a mini
Ebben a kezdő jelenetben több dolog is történik:
\begin{itemize}
    \item A hátteret egy később bemutatott algoritmus generálja újra és újra.
    \item A képernyőn piros háromszögek csúsznak balra, ahol egy sárga kör található. Ez a sárga kör a felhasználót reprezentálja.
    \item A háromszögek fölött egy megnevezés található. Ez a program során használt jelenetek nevei. Ha nekiütközünk valamelyik háromszögnek, akkor az adott jelentre ugrunk át.
\end{itemize}
A mini-játék során a SPACE billentyűvel tudunk ugrani. Ha nem szeretnénk ezzel a jelenettel bajlódni, akkor az ESC billentyűvel kiléphetünk a fő menübe, ahonnan egyszerűbben navigálhatunk tovább.\\
A jelenetekről részletes útmatás elérhető az a \ref{Utmutato} fejezetben.
