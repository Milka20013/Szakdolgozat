\chapter{Színterek}
A dolgozatom lényege a színterek generálása, generálási folyamataiknak bemutatása. Ezáltal
érdemesnek tartom a színtér fogalmát egy kicsit részletesebben bemutatni.\\
Önmagában a színtér alatt egy adott helyszínt, teret, hátteret értek a dolgozatom során. A színtér
magába foglalja az abban elhelyezett elemek, színek, textúrák és effektek összeségét. Ez lényegében
bármi lehet. Egy videójátékban vagy animációban akármit fel lehet használni, mint színtér.\\
Gondoljunk bele, hogy egy szájvizet reklámozó animációs filmben mit láthatunk a képernyőn. El lehet
képzelni egy fogászati rendelőt, egy fürdőszobát, de még egy emberi szájat vagy akár egy szimpla
fogat is. Nyilván az, hogy hol és miként játszódik az adott virtuális világ az nagyban befolyásolja az üzenetet, amit sugall a felhasználók felé.\\
Még egy ilyen példa a Grounded nevű videójáték. A játék
története szerint összezsugorodott a játékos és egy hagyományos kertes ház udvarában ébredt. Ezen
példában a színtér maga az udvar, de a perspektíva speciálisabb, ugyanis a fűszálak akkorák a
játékoshoz képest, mint a fák az emberekhez képest a valóvilágban.\\
Az idáig említett példákban a színteret valószínűleg kézzel, fix elem elhelyezésekkel készítették el.
Ugyan a való világban ezt nem igazán lehet megkönnyíteni, hiszen egy szobát nem lehet kattintásra
átrendezni, a virtuális világban szerencsére egyszerűbb a helyzet. Ez persze nem jelenti azt, hogy több
száz, akár ezer elem elhelyezése ne lenne nehéz feladat. Ennek könnyítésére és sokszínűségének
bővítésére szolgál a színtér procedurális generálása.

\section{Színterek csoportosítása}
A színtereket többféle szempont alapján tudjuk csoportosítani. Talán a legalapvetőbb szempont, hogy a színtér valódi, vagy csak virtuális. A valódi színterekkel az életünk mindegy egyes másodpercében találkozunk, hiszen a szobánk, az utcák, minden, minket körülvevő környezet egy valódi színtér. Virtuális színterekkel is találkozhatunk nap mint nap. Mikor megnézünk egy filmet, játszunk a telefonunkon, virtuális színterek tömkelege tárul elénk. Ezen esetekben már fel is merül egy újabb csoportosítási szempont: a dimenziók száma.\\
A környezetünkben már megszoktuk, hogy lényegében minden háromdimenziós. Még egy papírlap is rendelkezik vastagsággal, sőt, extrém példaként egy atom is háromdimenziós. Viszont nem egy valóságtól elrugaszkodó koncepció a kétdimenzió, hiszen sok olyan elem van a világban, aminek vastagságát elhanyagolhatjuk. Ilyen a tábla felszínére krétával rajzolt pálcikaember is, aminek ugyan vastagsága nem nulla, mégis annak vesszük. Ebben az esetben a tábla egy kétdimenziós színtér, a pálcikaemberünk pedig egy kétdimenziós elem rajta.\\
A virtuális színterek esetében a két és háromdimenziós elemek gyakran keverednek. Jellemző az, hogy a térben mélységében máshol helyezkednek el az elemek, sőt, háromdimenziós elemeket is tartalmazhat, csak szimplán egy síkot látunk a kamerán keresztül. Szokás ezeket a játékokat 2.5 dimenziósnak hívni, mert egyik kategóriába sem esik igazán. \\
Érdekesség, hogy léteznek próbálkozások a magasabb dimenziós virtuális színterek bemutatására is, főleg játékok terén \#KÉP. Ez az absztrakció már a dolgozatomhoz túl bonyolult, így ezekre csak említésképp hivatkoztam.

\section{Virtuális színterek készítése}
A készítés nulladik fázisa az, hogy kitaláljuk mi a kontextusa a színtérnek. Maga a
kontextus meghatározza, hogy milyen módszereket használjunk, illetve, hogy milyen célt kell
elérnünk. Egy végtelenített játék, vagy egy fél perces animáció színterét valószínűleg máshogyan kell
eltervezni, illetve a készítés is más módszerekkel fog lezajlani.\\
Az első fázis a tervezés. Érdemes ilyenkor meghatározni a stílust, a színtér lényegét, a benne szereplő elemek szerepét. Segítheti a munkát, ha már ebben a fázisban elképzeljük, hogy milyen hatást
szeretnénk kiváltani a felhasználókból. Ezek a készítés során rugalmasan változhatnak, viszont egy
kiinduló pont mindig jól jön.\\
Második fázis az elrendezés dizájn, az általános paraméterek meghatározása. Ilyen paraméterek
például a színtér nagysága, alakja, az elemek sűrűsége stb. Dizájn szempontjából érdemes átfogó
rajzokat készíteni mind az elemekről, mind magáról az elgondolt végeredményről is.\\
A harmadik fázisban érdemes elkészíteni az alapokat. Ilyen alap például a talaj, amire később
építhetünk. Ha egy tájat szeretnénk készíteni, akkor a természeti adottságokat, mint a hegyek, folyók
nagyságát, körülbelüli helyzetét érdemes ilyenkor nagyjából lefektetni. Egy házban játszódó jelenet
esetén a szobákat és a ház alaprajzát lehet elkészíteni.\\
Negyedik fázis az elemek elhelyezése, ez az egyik kulcsfontosságú lépés. Meghatározhatjuk, hogy
mely komponensek legyenek a fókuszpontban, mit lásson először a felhasználó. Az egyes elemek
közötti kölcsönhatásokat adhatjuk meg.\\
Az ötödik fázis a részletek elkészítése. A fények, effektek, alig látható kis részletek elhelyezése. Ebben a fázisban már közel késznek vehető a színtér. Ekkor következhet a tesztelés, a javítás, és
véglegesítés.\\
A dolgozatom a harmadik és negyedik lépések automatizálásával foglalkozik.

\section{A procedurális generálásról}
A generálás elsősorban a változatosságot hivatott elérni. Ilyen megoldásokat alkalmazó területeken
viszont felmerül egy ezzel ellentmondásban lévő probléma: a generált színterek egysíkúnak tűnik. Ha
nincs véletlenszerűség beleépítve az algoritmusban, akkor túl kiszámítható, ha túl sok van benne,
akkor pedig túl kiszámíthatatlan lesz a végeredmény. Még ha meg is találjuk az arany középutat,
akkor is hiányosnak érezhetünk egyes részeket. \# Kate Compton hivatkozás\\
Ennek ellenére előnyei is vannak a generálásoknak. Nem feltétlenül kell egy legenerált színteret
azonnali végeredményként felfogni. A készítés során egyszerűen gyárthatunk prototípusokat, amiket
később kiegészíthetünk, érdekessé tehetünk sajátkezűleg. Olyan munkakörnyezetben, ahol nincsen
kifejezetten ember arra, hogy színtereket készítsen, nagyon sokat segíthet, ha számtalan koncepciót
tudnak pár gombnyomással előállítani.\\
A procedurális generálás nem egy varázslat, ami adott helyzeteket azonnal megold. Inkább egy
eszköz, amit megfelelően kell használni. Ha jól kihasználjuk a benne rejlő lehetőségeket, akkor
minimális idő befektetésével nagy mennyiségű, minőségi tartalmakat készíthetünk.